\documentclass[IEEEtran,letterpaper,10pt,titlepage,fleqn,draftclsnofoot,onecolumn]{article}
%notitlepage vs titlepage
%fleqn left align

%\usepackage{nopageno} %no page numbers
\usepackage{indentfirst}
\usepackage{alltt}                                           
\usepackage{float}
\usepackage{color}
\usepackage{url}

\usepackage{graphicx}                                        
\usepackage{amssymb}                                         
\usepackage{amsmath}                                         
\usepackage{amsthm}                                          

\usepackage{balance}
%\usepackage[TABBOTCAP, tight]{subfigure}
\usepackage{enumitem}
\usepackage{pstricks, pst-node}
\usepackage{geometry}
\usepackage{hyperref}
\usepackage{textcomp}
\usepackage{listings}
%allows for code snipets

\geometry{textheight=9.5in, textwidth=7in} 

\newcommand{\cred}[1]{{\color{red}#1}} %think function call, changes text to red
\newcommand{\cblue}[1]{{\color{blue}#1}} %text to blue
\definecolor{dkgreen}{rgb}{0,0.6,0}
\definecolor{gray}{rgb}{0.5,0.5,0.5}
\definecolor{mauve}{rgb}{0.58,0,0.82}
\lstset{frame=tb,
  language=c,
  aboveskip=3mm,
  belowskip=3mm,
  showstringspaces=false,
  columns=flexible,
  basicstyle={\small\ttfamily},
  numbers=none,
  numberstyle=\tiny\color{gray},
  keywordstyle=\color{blue},
  commentstyle=\color{dkgreen},
  stringstyle=\color{mauve},
  breaklines=true,
  breakatwhitespace=true,
  tabsize=3
}

\def\name{Brandon Ellis, Jiayu Han, and Jack Neff}
\def\class{CS 461 }
\def\assignment{Problem Statement}

%PDF Properties
\hypersetup{
  colorlinks = true,
  urlcolor = black,
  pdfauthor = {\name},
  pdfkeywords = {cs461 ``Senior Capstone' Problem Statement},
  pdftitle = {\class \assignment},
  pdfsubject = {\class \assignment},
  pdfpagemode = UseNone
}

\begin{document}
%TitlePage
\begin{titlepage}
	\begin{center}
		\vspace*{1cm}
		
		\huge
		\textbf{Green Smart Gardening System: Problem Statement}
        
        \vspace{1.5cm}
        
		\large
        \textbf{Brandon Ellis, Jiayu Han, and Jack Neff}
		
		\vspace{5cm}
		
		\normalsize
		This paper covers the challenges and solutions proposed in Green Smart Gardening System, an analytical smart gardening device. First, this paper will address the problem that our created device will seek to solve. It will then discuss the proposed solution, broken down into the device’s inputs, outputs, hardware, and analysis. Finally, this paper will conclude with some estimated performance metrics which will define the completion of our device.
		
		\vfill
        
		\large
        CS 461\\
        Fall Term\\
    \end{center}
\end{titlepage}

\section{Description of Problem}

As the use of smart home technology increases so has the interest in smart gardening. We plan to create a device that can simplify some aspects of gardening and provide the user with more in-depth analysis of their garden. The device would enable the user to contrast the health of their plant with environmental statistics, making informed care of their garden that much easier. The original use case for this device was based around the wine industry in Southern Oregon. The creation of a smart gardening device, specifically in this growing sector, would aid in the growth of both businesses and the plants that sustain them. 

\vspace{5mm}

Digital gardening systems are nothing new. Timed irrigation, lighting, and humidity systems are commonplace today, but they are heavily reliant on user input, and are not capable of monitoring plant growth or health. If an inexperienced user sets the timers wrong, or fails to account for a change in environmental factors, this type of automated gardening system will fail. In our design, we want to build something that an inexperienced user will be comfortable using, and that will provide gardeners and farmers with a better understanding of important factors in the growth process. If plants are wilting or dying, at the very least our system will alert the user, prompting them to change the settings. Ideally, we will be able to design a system that self-corrects, totally minimizing the responsibility of the user. As it grows plants, our system will report environmental factors, water use, and plant growth and provide the user with data and graphics relating these factors, to give them a better understanding of their garden. This will involve devising a way to measure plant growth, and writing algorithms that analyze and plot relevant data.

\vspace{5mm}

When we shift responsibility from the user to the system, we need to find a reliable and efficient way to power that system. As far as efficiency goes, this means calculating the energy requirements of our proposed system, and finding an appropriate battery that isn’t unnecessarily large or expensive. For redundancy, we will also install solar panels to power the array on nice days. Solar panels will be spotty for a certain period of time in the colder months, so we will need to select a battery that lasts longer than that period. 

\vspace{5mm}

\section{Proposed Solution}

What this project is aiming to accomplish is the creation of an environmental monitoring device. While a specific microcontroller has yet to be narrowed down, said device has a number of specific necessities. I will be breaking these specifics down into four portions: input, output, hardware, and analytics.

\subsection{Input}

To fulfill the gardening aspect of this device, it tracks several environmental conditions. For soil, the device will measure the moisture level in the soil as well as the ph. Additionally, a single hardware addition will capture the ambient air temperature, the air humidity, and air pressure. There is also the possibility of adding the control or information from an automated watering system to the devices network.

\subsection{Output}

Once all of the above inputs have been recorded, the device needs to send them somewhere useful. There is one location planned currently for the data to go, but there are several routes that the data can take from it. First the data will always go to a computer for analysis. Once the data has been quantified, a couple different things could occur depending on our group’s initial and stretch goals. The first is that the data could be sent wirelessly to a smartphone or another smart device. From there, the user could reference a more viewable representation of the data at their leisure. The second tags onto outputting the data for viewing, a current stretch goal is the creation of a viewing model that would compare specifics plants’ health against noted environmental factors. Lastly, the computer would be able to interface with an existing smart home system to control environmental factors. These could include the ambient temperature, lighting, and sprinklers.

\clearpage

\subsection{Hardware}

To achieve this the microcontroller will have to be augmented. First, the inputs require the device communicates with several sensors. Specifically, we are currently looking at the Adafruit BME280 I2C, for soil temperature, humidity, and pressure; DFRobot Gravity, for soil moisture; and EZO pH Circuit, for the pH of the soil.

\vspace{5mm}

Once the data has been collected, the microcontroller needs to output it. Depending on the direction of the project, the output could be done a couple of different ways. There will be a guaranteed connection from the device to a computer for the analysis of data. This will most likely be achieved by a wireless connection between the two, though it could be hardwired. When the data has been analyzed by the computer, there is a stretch goal that would connect said computer wirelessly to a smartphone or other internet enabled device. All of these result in the microcontroller needing at least one possible form of wireless output.

\vspace{5mm}

The last hardware requirement lies in how the device is powered. A key requirement is that the microcontroller be powered through green methods, with the preferred form being solar. This will mean that the device will need to have a solar panel that into the microcontroller. The current battery pack being eyed is the MSP430. To fulfil all requirements this charger may need to be expanded, as it has the possibility to add additional solar panels. It does have a rechargeable battery contained within for storing the charge during lower light.

\subsection{Analytics}

This bring us to the last piece of functionality, the analysis of data. There are two separate portions to this. The first analysis is done purely on the recorded data. After the microcontroller outputs raw data to a computer, an analytical model is required to be viewable. This model provide access all of outputted data. It does not contrast this information, instead simply making it accessible for the user to reference. The second form of analysis is currently a stretch goal. This analysis is on the aforementioned data once it has been outputted to the computer, but it does contrast the information against the general health and growth of a specific plant. At this point, how the health and growth are measured is undefined. Most likely these values will be inputted by the user, based on their insights and measurements. If we are able, we could use a common data mining algorithm like SVM or k-means to figure out what environmental configurations help or harm a plant, and what configurations produce the most growth. 

\vspace{5mm}

\section{Performance Metrics}

For this project to be considered complete, there are several criteria. First, the microcontroller must receive valid, continuous within the realm of the input’s rate, string of data from each of its inputs. This will result in the device collecting data, in designated intervals, on the soil, temperature, and air. Second, the microcontroller must be powered through solar energy. This will result in a green device, which is a key requirement for this project. Third, all captured data must be fed to a computer, and said computer must quantify all data for the user’s viewing. When these three tasks have been completed, the base specification of the device will be achieved. However there is one key aspect of this device that is overarching. This is the assurance that the microcontroller and all of its additions can continue to perform under varying climates. Specifically, the original use case that was mentioned above, was for the creation of a device to aid the wine industry in Southern Oregon. Such a device would have to monitor all of aforementioned environmental aspects, as well as work in anything from sun to snow to rain.

\vspace{5mm}

To reach the full stretch goals for the project two more requirements must be met. The first is that once the data has reached the computer for analysis, it can be transmitted wirelessly for viewing on another device. The second stretch goal is the creation of an analytical model for comparing the health of a plant with the data received from the microcontroller. The health of the plant will, at this time, be defined by the user. This information would be viewed at either the main computer or sent wirelessly to a device. 

\end{document}
